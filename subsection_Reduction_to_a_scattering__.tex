\subsection{Reduction to a scattering problem}
To evaluate this integral, $k(x)$ should, in principle, come from the (complex) dispersion relation of the crystal (or some model thereof).  In our approach, we will properly evaluate $k(x)$; however, it's worth noting the clever trick which Sze leverages to ignore the details of the dispersion and map the solution onto an introductory-level scattering problem.  In explaining this point, we will consider only electrons with no transverse momentum, and we briefly restore the assumption of uniform electric field, $\xi(x)=\xi=E_G/q(x_c-x_v)$.

With the assumption of parabolic bands (both of effective mass $m^*$) for small $k$, one can  analytically continue the dispersions $E(k)=E_v(x)-\hbar^2k^2/2m^*$, $E(k)=E_c(x)+\hbar^2k^2/2m^*$ onto imaginary $k$ to find valid solutions in the gap.  (Note that this only holds near to the respective band edges because of the parabolicity assumption.)  So, near $x_v$, we have $E=E_V(x)+\hbar^2|k|^2/2m^*$.  Since, at the start of the tunnel region, $E_V(x_v)=E$, and the slope of $E_V$ is given by $\xi$, $E=E-\xi(x-x_V)+\hbar^2|k|^2/2m^*$, ie
$$ik(x)=\sqrt{2m^*q\xi(x-x_v)/\hbar^2}$$
The above is formally the same as the classic dispersion of an particle of energy $E$ in a barrier $U(x)$, where $U(x)-E=q\xi(x-x_v)$.  Similarly, near $x_c$, we have $ik=\sqrt{2m^*q\xi(x_c-x)/\hbar^2}$, which is equivalent to a barrier $U(x)-E=q\xi(x-x_c)$.  When the electron is deep into the tunneling region, the dispersion is more complicated but Sze just interpolates the simplest algebraic functional form for a barrier $U(x)$ which reproduces the correct dispersion near the edges.  That is the quadratic:
$$U(x)-E=\frac{(E_G/2)^2-(q\xi x)^2}{E_G}$$
So, in Sze's approximation, the transmission coefficient for interband tunneling is just given by the transmission coefficient of a particle with a familiar parabolic dispersion (and mass $m^*$) through the above potential barrier.  Conveniently, this coefficient does not even depend on the energy of the tunneling electron (within the WKB approximation, for uniform fields, and ignoring transverse momentum).