\subsection{Reduction to a scattering problem}
To evaluate this integral, $k(x)$ should, in principle, come from the (complex) dispersion relation of the crystal (or some model thereof).  In our approach, we will properly do so; however, it's worth noting the clever trick which Sze leverages to ignore the details of the dispersion and map the solution onto an introductory-level scattering problem.  (In explaining this point, we briefly restore the assumption of uniform electric field $\xi(x)=\xi$, and will consider only electrons with no transverse momentum.)

With the assumption of parabolic bands (both of effective mass $m$) for small $k$, one can just analytically continue the dispersions $E(k)=E_V(x)-\hbar^2k^2/2m$, $E(k)=E_C(x)+\hbar^2k^2/2m$ onto imaginary $k$ to find valid solutions in the gap.  (Note that this only holds near to the respective band edges because of the parabolicity assumption.)  So, near $x_V$, we have $E(k)=E_V(x)+\hbar^2|k|^2/2m=E-\xi(x-x_V)+\hbar^2|k|^2/2m$