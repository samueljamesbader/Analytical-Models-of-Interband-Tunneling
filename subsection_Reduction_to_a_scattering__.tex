\subsection{Reduction to a scattering problem}
To evaluate this integral, $k(x)$ should, in principle, come from the (complex) dispersion relation of the crystal (or some model thereof).  In our approach, we will properly evaluate $k(x)$; however, it's worth noting the clever trick which Sze leverages to ignore the details of the dispersion and map the solution onto an introductory-level scattering problem.  In explaining this point, we briefly restore the assumption of uniform electric field $\xi(x)=\xi$, and will consider only electrons with no transverse momentum.  So $q\xi=E_G/(x_C-x_V)$

With the assumption of parabolic bands (both of effective mass $m^*$) for small $k$, one can just analytically continue the dispersions $E(k)=E_V(x)-\hbar^2k^2/2m^*$, $E(k)=E_C(x)+\hbar^2k^2/2m^*$ onto imaginary $k$ to find valid solutions in the gap.  (Note that this only holds near to the respective band edges because of the parabolicity assumption.)  So, near $x_V$, we have $E=E_V(x)+\hbar^2|k|^2/2m^*=E-\xi(x-x_V)+\hbar^2|k|^2/2m^*$, ie $ik=\sqrt{2mq\xi(x-x_V)/\hbar^2}$.  That is the same as the regular dispersion of an particle in a barrier $U(x)$ such that $U(x)-E=q\xi(x-x_V)$.  Similarly, near $x_C$, we have $ik=\sqrt{2m^*q\xi(x_C-x)/\hbar^2}$, which is equivalent to a barrier $U(x)-E=q\xi(x-x_C)$.  When the electron is deep into the tunneling region, the dispersion is more complicated but Sze just interpolates the simplest algebraic functional form for a barrier $U(x)$ which reproduces the correct dispersion near the edges.  That is the quadratic:
$$U(x)-E=\frac{(E_G/2)^2-(q\xi x)^2}{E_G}$$
So, in Sze's approximation, the transmission coefficient for interband tunneling is just given by the transmission coefficient of a particle with a familiar parabolic dispersion (and mass $m^*$) through the above potential barrier.