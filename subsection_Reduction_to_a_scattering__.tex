\subsection{Reduction to a scattering problem}
To evaluate this integral, $k(x)$ should, in principle, come from the (complex) dispersion relation of the crystal, or some model thereof (eg \cite{Guan_2011}).  In our approach, we will properly evaluate $k(x)$; however, it's worth noting the clever trick which Sze leverages to ignore the details of the dispersion and map the solution onto an introductory-level scattering problem.  In explaining this point, we will consider only electrons with no transverse momentum, and we briefly restore the assumption of uniform electric field, $\xi(x)=\xi=E_G/q(x_c-x_v)$.

With the assumption of parabolic bands (both of effective mass $m^*$) for small $k$, one can  analytically continue the dispersions $E(k)=E_v(x)-\hbar^2k^2/2m^*$, $E(k)=E_c(x)+\hbar^2k^2/2m^*$ onto imaginary $k$ to find valid solutions in the gap.  (Note that this only holds near to the respective band edges because of the parabolicity assumption.)  So, near $x_v$, we have $E=E_V(x)+\hbar^2|k|^2/2m^*$.  Since, at the start of the tunnel region, $E_V(x_v)=E$, and the slope of $E_V$ is given by $\xi$, $E_V=E-q\xi(x-x_V)$, and
$$ik(x)=\sqrt{2m^*q\xi(x-x_v)/\hbar^2}$$
The above is formally the same as the classic dispersion of an particle of energy $E$ in a barrier $U(x)$, where $U(x)-E=q\xi(x-x_v)$.  Playing the same game near $x_c$, we find an effective barrier $U(x)-E=q\xi(x_c-x)$.  When the electron is deep into the tunneling region, the dispersion is of course more complicated.  Nonetheless, Sze interpolates the simplest algebraic functional form for a barrier $U(x)$ which fits to the above limits.  That is the quadratic:
$$U(x)-E=\frac{(E_G/2)^2-(q\xi x)^2}{E_G}$$
where the zero of $x$ has been implicitly set to the mean of $x_c$ and $x_v$.  In Sze's approximation, the transmission coefficient for interband tunneling is just given by the transmission coefficient of a particle with a familiar parabolic dispersion (and mass $m^*$) through the above parabolic potential barrier.  Conveniently, this coefficient does not even depend on the energy of the tunneling electron (within the WKB approximation, for uniform fields, and ignoring transverse momentum).

Allowing for transverse momentum can be shown to simply raise the parabolic barrier by $E_\perp=\hbar^2k_\perp^2/2m^*$.  Including this effect, and evaluating the WKB integral, we find
$$T(E,E_\perp)=\exp\left\{-\frac{\pi\sqrt{m^*}(E_G+2E_\perp)^{3/2}}{2\sqrt{2}q\hbar\xi}\right\}$$