\section{Sze: Reduction to a scattering problem}
Given the mathematical ease of Sze's method, it is the more natural place to begin.

In a semi-classical, effective mass/envelope function picture of electron transport in a crystal, the state is imagined to be a wavepacket with simultaneously well-defined and smoothly evolving position $x$ and crystal momentum $\hbar k$.  As the electron enters the classically forbidden region, the $x$-component of its wavevector passes through zero onto the imaginary axis; as the electron exits the region, its wavevector passes back through zero once more and returns to the real axis.  The energy $E$ of the tunneling electron is constant, but the bands $E_C(x)$, $E_V(X)$ bend throughout the device.  At any position $x$, one could then determine the electron's energy relative to changing the band edges, and then from the crystal dispersion relation, compute $k(x)$.

The WKB approximation provides a simple expression for how the amplitude of the wavefunction decays as it passes through the forbidden region, from which we compute the transmission coefficient
$$T(E)=\int_{x_v}^{x_c}dx\abs{k(x)}$$