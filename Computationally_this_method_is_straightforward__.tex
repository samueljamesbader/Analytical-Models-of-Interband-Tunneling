\textbf{HHHHHHHHHH}

Computationally, this method is straightforward to apply even when the assumptions underlying its simple analytical forms are removed.  Starting from a given band diagram, the procedure is as follows.  For each $\vec{k}$ in the valence band that could tunnel to the conduction band (while conserving energy and transverse momentum), find the start and end of the tunnel region (where $k_x=0$).  Then, choosing some determination of $k$ (either Sze's formula, or a band structure model) evaluate \ref{WKB}.  Once $T$ is known for all $k$'s, integrate over the band to find a tunnel current.



In the case of uniform fields and parabolic bands/two-band model, the above could be done analytically, since $T$ (above) does not depend on $E$.  However


In preparation for the final presentation, I will be studying the math to approximate this procedure for non-uniform fields and some realistic yet simple band model (likely of the form expressed in \cite{Guan_2011} or derived from \cite{Piprek}).