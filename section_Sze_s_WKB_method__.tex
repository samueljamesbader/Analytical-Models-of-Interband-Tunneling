\section{Sze's WKB method}
Given the mathematical ease of Sze's method, it is the more natural place to begin.  In a semi-classical, effective mass/envelope function picture of electron transport in a crystal, the state is imagined to be a wavepacket with simultaneously well-defined and smoothly evolving position $x$ and crystal momentum $\hbar k$.  The energy $E$ of the tunneling electron is constant, but the bands $E_C(x)$, $E_V(x)$ bend throughout the device.  At any position $x$, one could then determine the electron's energy relative to changing the band edges, and then, from the crystal dispersion relation, compute $k(x)$ for every position.

As the electron enters the classically forbidden region, the $x$-component of its wavevector passes through zero onto the imaginary axis and the wavepacket decays in amplitude.  As the electron exits the region, its wavevector passes back through zero once more and returns to the real axis.  Once $k(x)$ is known, the WKB approximation provides a simple expression for the transmission coefficient through a tunneling barrier:
$$T(E)=\exp\left\{-2\int_{x_v}^{x_c}dx|k(x)|\right\} \label{WKB}$$
where $x_v$ and $x_c$ are the valence and conduction-side boundaries of the tunnel region.