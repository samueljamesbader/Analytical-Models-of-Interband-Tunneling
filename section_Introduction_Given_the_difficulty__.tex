\section{Introduction}
Given the difficulty of creating good ohmic contacts to $n$-doped GaN, the nitrides community would benefit greatly from alternative structures which allow large currents to flow from $p$-contacts into $n$-GaN.  One such possibility is ``spiked'' $pn$ junctions in GaN, where the regrowth of the junction interface produces a built-in sheet charge.  This sheet charge notches the band diagram, forcing the high electric fields necessary for large tunneling currents.  Recent work has derived analytical approximations to the band diagrams of these devices, yet, due to the highly non-uniform fields, simple expressions of the tunneling currents in these devices remain an open problem.  Relating the design parameters, eg geometry and doping, of these interfaces to the tunneling current will be vital for rapid modeling if these structures are to find application in LEDs and more.

The initial theoretical studies of interband tunneling began with Keldysh \cite{Keldysh_1958} and Kane \cite{Kane_1960}.  In particular, the two-band model of Kane has proven an effective approximation for the study of many direct-gap semiconductors.  A major intuitive simplification was introduced by Sze \cite{Sze_2nd}, who approximated the problem as a classic tunneling wavefunction through a one-dimensional barrier, neglecting the details of the matrix element and band-structure.  Both methods assumed, however, a uniform electric field. Within Sze's picture, this assumption can be relaxed rather simply.  Many authors (eg \cite{Takayanagi_1991}, \cite{Tanaka_1994}) have since expanded Kane's model as well to non-uniform field, though at the cost of greater mathematical complexity.  Given the trade-offs of both Kane's and Sze's approaches, it will be useful to study the application of each model to the non-uniform field profile of the spiked $pn$ junction.
