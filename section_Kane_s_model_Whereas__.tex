\section{Kane's model}
Whereas Sze's approach could be described as a time-independent WKB method within the effective mass picture of transport, Kane instead applies time-dependent perturbation theory to the the Wannier equation
$$i\hbar\partial_t\psi_n(r,t)=E_n^0(k)\psi_n(r,t)+\phi(x)\psi_n(r,t)+\sum_{n\neq n'} W_{n\,n'}\psi_{n'}(r,t)$$
where the effect of the electric field $-\partial_x\phi(x)$ is two-fold.  First, by removing translation symmetry, it mixes the $k$-states within a band, and, secondly, it provides matrix elements $W_{n\neq n'}$ that mix bands together.  Kane solves for the wavefunctions with $W$ set to zero, and then applies Fermi's golden rule to determine the transition rate between bands.