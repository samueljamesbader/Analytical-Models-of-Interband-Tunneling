\section{Kane's model}
Whereas Sze's approach could be described as a time-independent WKB method within the effective mass picture of transport, Kane instead applies time-dependent perturbation theory to the the Wannier equation
$$i\hbar\partial_t\psi_n(r,t)=E_n^0(k)\psi_n(r,t)-e\phi(x)\psi_n(r,t)+\sum_{n\neq n'} W_{n\,n'}\psi_{n'}(r,t)$$
where the effect of the electric field $-\partial_x\phi(x)$ is two-fold.  First, by removing translation symmetry, it mixes the $k$-states within a band, and, secondly, it provides matrix elements $W_{n\neq n'}$ that mix bands together.  Kane solves for the wavefunctions with $W$ set to zero, and then applies Fermi's golden rule to determine the transition rate between bands.

In Kane's approach, the Wannier equation is expressed not as above, but rather in the crystal momentum basis.  The procedure is the same in principle, though expression in this basis provides an integral which Kane was able to approximate analytically via the method of steepest descent.  Nonetheless, use of the crystal momentum basis becomes burdensome for all but the most trivial field profiles (the position operator is represented by a derivative in $k$-space, so fields that are non-trivial functions of position become higher-order derivatives in the Wannier equation).  Recent generalizations of Kane's model to non-uniform fields thus prefer the familiar position representation \cite{Tanaka_1994}.

\subsection{Approach to generalize}
All the work discussed involving Kane's method of evaluating the interband tunneling has been within Kane's two-band $k\cdot p$ model of the band structure, which works will for many semiconductors (originally InSb), but less so for III-V's.  It would be a useful exercise to try expressions from larger models (ie the 4x4 $k\cdot p$ model which has been fitted to GaN \cite{Rinke_2008}) and see whether the mathematics can still be performed analytically by similar techniques, especially  in the context of Tanaka's more general formulation.