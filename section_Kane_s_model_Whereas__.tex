\section{Kane's model}
Whereas Sze's approach could be described as a time-independent WKB method, Kane instead applies time-dependent perturbation theory to the the Wannier equation
$$i\hbar\partial_t\psi_n(r,t)=E_n^0(k)\psi_n(r,t)+U(x)\psi_n(r,t)+\sum_{n\neq n'} W_{n\,n'}\psi_{n'}(r,t)$$
where the effects of the electric field is two-fold.  First, it mixes the $k$-states within a band, and, secondly, it provides 