\section{GaN Bandstructure}
Sze simplifies the problem by using a parabola with curvature set by the effective mass for the dispersion, but one really should use the material-appropriate imaginary bandstructure in the gap, as shown in \ref{fig:imagBS}.  In order have a usable form for the bandstructure, I've taken the 4x4 $k\cdot p$ Hamiltonian from \cite{Rinke_2008}, and specified it to the $\hat{z}$ direction, which removes most of the off-diagonal matrix elements.  In fact, along the $z$-axis, the Hamiltonian can be reduced to an effective two-band model (similar to but more general than \cite{Kane_1960}.
\[
  H=\begin{pmatrix}
    E_g'+\frac{\hbar^2k_z^2}{2m_e} & iP_1k_z\\
    -iP_1k_z & \frac{\hbar^2k_z^2}{2m_h} \\
  \end{pmatrix}
\]
\[
  E_g'\equiv E_g+\Delta_{CR}, \quad \frac{1}{m_e}\equiv\frac{1}{m_0}+\frac{2A_1'}{\hbar^2}, \quad \frac{1}{m_h}\equiv\frac{1}{m_0}+\frac{2L_2'}{\hbar^2}, \quad
\frac{1}{m_r}\equiv\frac{1}{m_e}+\frac{1}{m_h}
\]




Computationally, this method is straightforward to apply even when the assumptions underlying its simple analytical forms are removed.  Starting from a given band diagram, the procedure is as follows.  For each $\vec{k}$ in the valence band that could tunnel to the conduction band (while conserving energy and transverse momentum), find the start and end of the tunnel region (where $k_x=0$).  Then, choosing some determination of $k$ (either Sze's formula, or a band structure model) evaluate \ref{WKB}.  Once $T$ is known for all $k$'s, integrate over the band to find a tunnel current.



In the case of uniform fields and parabolic bands/two-band model, the above could be done analytically, since $T$ (above) does not depend on $E$.  However


In preparation for the final presentation, I will be studying the math to approximate this procedure for non-uniform fields and some realistic yet simple band model (likely of the form expressed in \cite{Guan_2011} or derived from \cite{Piprek}).